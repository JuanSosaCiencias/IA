\textbf{Menciona dos diferencias entre la función de rendimiento y el programa del agente (1 pt.).} \vspace{.3cm}

Como ya sabemos, la función de rendimiento busca evaluar qué tan bien un agente cumple con sus objetivos en su entorno, mientras que el programa del agente se encarga de implementar este comportamiento. Por lo tanto, esta primera diferencia puede compararse con la relación entre teoría y práctica.  

De la misma manera, es evidente que la implementación de la función de rendimiento es independiente del programa del agente; en este sentido, la función representa el \textit{qué} y el programa el \textit{cómo}.  

Finalmente, cabe mencionar que es relativamente más fácil cambiar el programa del agente para mejorar su desempeño, mientras que modificar la función de rendimiento suele ser mucho más complicado. \vspace{.3cm}
