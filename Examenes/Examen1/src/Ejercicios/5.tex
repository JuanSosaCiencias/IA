\textbf{Investiga en qué informe se declara el fracaso del programa de traducción de máquina, y explica brevemente las razones que en él se esbozan. Cita tus fuentes (1 pt.).}

El informe al que se refiere la pregunta es el elaborado por el \textit{Automatic Language Processing Advisory Committee} (ALPAC) en 1966, donde se revisaron los avances en este tipo de inteligencia artificial y se concluyó que los resultados no eran satisfactorios, recomendándose redirigir el enfoque y los recursos hacia otras áreas de investigación. \vspace{.3cm}

De manera sencilla, el programa fracasó porque no era capaz de generar traducciones de buena calidad, especialmente en comparación con aquellas realizadas por traductores profesionales. Además, se mencionaron grandes limitaciones tanto en hardware como en software de la época, y se destacó que el lenguaje natural es demasiado ambiguo y complejo. En esencia, al comparar los resultados obtenidos con la inversión realizada, el proyecto resultaba insostenible. Este fracaso marcó el inicio del primer "invierno de la IA", lo que redujo significativamente el financiamiento para la investigación en el área, además de que los métodos disponibles no eran lo suficientemente avanzados. \vspace{.3cm}

\cite{alpac1966machine}
