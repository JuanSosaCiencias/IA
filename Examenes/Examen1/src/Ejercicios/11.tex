\textbf{Explica la noción de aprendizaje para agentes (1 pt.).} \vspace{.3cm}

El aprendizaje, no es mas que la capacidad de adaptarse y mejorar sus acciones respecto a su entorno de trabajo, para ello el agente debe obtener experiencia interactuando con su ambiente mediante sus perceptores.

Para procesar y poder \textit{aprender} como tal, se deben acoplar una forma de los siguientes elementos:
\begin{itemize}
    \item \textbf{Critica:} El conocido feedback, es el mecanismo preestablecido que es capaz de darle al agente una manera de saber si sus acciones son las deseadas.
    \item \textbf{Elemento de aprendizaje:} Esta es la parte interesante, pues permite extraer partrones y ajustar la manera en la que el agente toma decisiones en funcion de los resultados obtenidos (critica).
    \item \textbf{Elemento de eficacia:} Esta parte es la que permite al agente tomar la mejor decision para cumplir su objetivo o satisfacer su funcion de utilidad.
    \item \textbf{Generador de problemas:} Maso menos como tu ex (chiste), el generador de problemas literalmente explora sitauciones mas alla de la tarea inmediata para poder explorar situaciones sin tener que exponer el agente fisico a ellas, asi ampliando el conocimiento y evitando estancamiento. 
\end{itemize}

\cite{SotoAstorga2025} \cite{russell2020artificial}