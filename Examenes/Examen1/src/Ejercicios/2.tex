\textbf{Explica los tres paradigmas principales de la I.A., así como sus características principales (1 pt.).}

\begin{itemize}
    \item \textbf{Simbólico}
    
    En esencia, este enfoque busca conseguir "inteligencia" mediante el uso de símbolos y reglas. Es útil si queremos hacer cosas más determinísticas o más estructuradas. \vspace{.3cm}

    Es un enfoque más limitado en términos de escalabilidad, pues se tiene que, de cierta forma, saber de manera previa lo que se quiere que el sistema sepa. \vspace{.3cm}
    
    \cite{russell2020artificial}

    \item \textbf{Estadístico}
    
    Como su nombre lo indica, este paradigma se apoya en métodos probabilísticos y estadísticos, como la regresión, para intentar corregir o determinar la incertidumbre. Incluye métodos como los de aprendizaje supervisado, no supervisado y de refuerzo. \vspace{.3cm}

    En sí, son bastante útiles para detectar patrones en grandes cantidades de datos y se aplican en modelos económicos o en los conocidos algoritmos de redes sociales. \vspace{.3cm}

    En esencia, este enfoque presenta dos problemas principales. El primero, y el más grande (no solo para este tipo de IA), es la cantidad y calidad de los datos; al tratarse de modelos diseñados para procesar grandes volúmenes de información, conseguir datos y verificar que sean válidos es complicado. El segundo problema es el de la caja negra: a diferencia del enfoque anterior, en el que los procesos son relativamente más simples de entender y simular, en este paradigma, al no ser necesario mostrar o explicar cada paso sino solo el resultado y el modelo, se vuelve bastante más complejo comprender su funcionamiento. \vspace{.3cm}

    \cite{bishop2006pattern}

    \item \textbf{Neuronal}
    
    Finalmente, tenemos el paradigma neuronal. Como su nombre lo indica, esta corriente se basa en el funcionamiento de los cerebros orgánicos, especialmente el de los humanos; en su núcleo se plantea la idea de computar a través de una red de unidades independientes distribuidas, que reciben una serie de señales y generan otra serie de señales tras procesarlas. \vspace{.3cm}

    Además de las unidades, denominadas neuronas, existen capas de entrada, ocultas y de salida, que son agrupaciones de neuronas con comportamientos específicos. Una ventaja de este enfoque es que pierde gran parte de la estructura rígida que requieren los otros dos, volviéndose más flexible y permitiendo enfocarse en las entradas y salidas. Esto, a su vez, implica que nuevamente surge el problema de la caja negra, y en este caso, aún más marcado. \vspace{.3cm}

    Este enfoque es el más popular actualmente, a mi parecer, por su gran escalabilidad; además, estos modelos son capaces de tratar con datos no estructurados, lo que los hace mucho mejores en tareas de traducción o procesamiento de imágenes. \vspace{.3cm}

    \cite{goodfellow2016deep}
\end{itemize}

Finalmente, me gusta agregar que, aunque son tres enfoques diferentes, en la práctica es muy común utilizar una combinación para aprovechar las ventajas de cada uno y compensar las limitaciones de los otros.
