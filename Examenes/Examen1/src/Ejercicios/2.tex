\textbf{Explica los tres paradigmas principales de la I.A., así como sus características principales (1 pt.).}

\begin{itemize}
    \item \textbf{Simbólico}
    
    En escencia este enfoque busca conseguir "inteligencia" mediante el uso de simbolos y reglas. Es útil si queremos hacer cosas mas deterministicas o mas estructurada. \vspace{.3cm}

    Es un enfoque mas limitado en terminos de escalabilidad, pues se tienen que de cierta forma saber de manera previa lo que se quiere que el sistema sepa. \vspace{.3cm}
    
    \cite{russell2020artificial}

    \item \textbf{Estadístico}
    
    Como su nombre lo dice, este paradigma se apoya de metodos probabilisticos y estadisticos como la regresión para intentar corregir o determinar la incertidumbre, incluyen metodos como los de aprendizaje supervisado, no supervisado y de refuerzo. \vspace{.3cm}

    En si son bastante utiles para ver patrones en grandes cantidades de datos y son lo que se implementa para modelos economicos o los conocidos algoritmos de redes sociales.\vspace{.3cm}

    En escencia este enfoque tiene 2 problemas principales, el primero y el mas grande (no solo aplica a este tipo de IA) es la cantidad y calidad de los datos, en si, al ser modelos hechos para procesar muchos datos, conseguirlos y ver que sean validos es complicado. El segundo problema es el de la caja negra, a diferencia del enfoque anterior en donde las cosas son relativamente mas simples de entender y simular, en este paradigma, al no necesitar mostrar o explicar los pasos si no mas bien el resultado y el modelo, se vuelve bastante mas complicado entender su funcionamiento. \vspace{.3cm}

    \cite{bishop2006pattern}

    \item \textbf{Neuronal}
    
    Finalmente, tenemos el paradigma neuronal, como su nombre lo indica, esta corriente se basa de como funcionan los cerebros organicos, especialmente el de los humanos; en su centro tiene la idea de computar a traves de una red de unidades independientes distribuídas que reciben una serie de señales y sacan otra serie de señales tras procesarlas. \vspace{.3cm}

    Ademas de las unidades que llaman neuronas, existen capas de entrada, ocultas y de salida, que son agrupaciónes de neuronas con comportamientos especificos. Una parte buena de este enfoque es que pierde gran parte de la estrucura que los otros 2 necesitan, volviendose mas flexible y permite enfocarse mas en las entradas y salidas, lo que a su vez significa que otra vez tenemos el problema de la caja negra y esta vez aun mas intenso. \vspace{.3cm}

    Este enfoque es el mas popular actualmente, a mi parecer por su gran escalabilidad, ademas, estos modelos son capaces de tratar con datos no estructurados, haciendolos mucho mejores en tareas de traduccion o de procesamiento de imágenes.\vspace{.3cm}

    \cite{goodfellow2016deep}
\end{itemize}

Finalmente, me gusta agregar que aunque son 3 enfoques diferentes, en la práctica es muy commun utilizar una combinación para aprovechar las ventajas de cada uno y cubrir lo que se puede de los otros.