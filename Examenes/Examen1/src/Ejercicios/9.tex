\textbf{Menciona tres categorías en las que podemos clasificar los entornos de trabajo (0.5 pt.).} \vspace{.3cm}

Las dimensiones que~\cite{SotoAstorga2025} menciona para categorizar los entornos según sus propiedades son:

\begin{itemize}
    \item \textbf{Observabilidad:} Se refiere a la cantidad de información que el agente puede obtener de su entorno mediante sus sensores. 
    \item \textbf{Cantidad de agentes:} Considera si existen otros agentes en el ambiente y cómo interactúan con nuestro agente.
    \item \textbf{Determinismo:} Un entorno es determinista si las acciones del agente, junto con el estado previo, determinan completamente la evolución del ambiente. De lo contrario, el entorno es estocástico o estratégico si los cambios dependen de otros agentes.
    \item \textbf{Episodicidad:} Un entorno es episódico si cada decisión puede considerarse de manera independiente, mientras que es secuencial cuando las acciones presentes afectan las decisiones futuras.
    \item \textbf{Estaticidad:} Evalúa qué tanto cambia el entorno mientras el agente ejecuta sus acciones. 
    \item \textbf{Continuidad:} Distingue si las acciones y el entorno evolucionan en intervalos de tiempo preestablecidos (discreto) o si se consideran continuos, como en el mundo real.
\end{itemize}
