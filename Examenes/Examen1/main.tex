\documentclass[11pt,letterpaper]{article}
\usepackage[utf8]{inputenc}

%----- Configuración del estilo del documento------%
\usepackage{epsfig,graphicx}
\usepackage[left=2cm,right=2cm,top=1.8cm,bottom=2.3cm]{geometry}
\usepackage{fancyhdr}
\usepackage{lastpage}

\usepackage{xcolor}
\usepackage{soul}
\newcommand{\mathcolorbox}[2]{\colorbox{#1}{$\displaystyle #2$}}

\usepackage{float}

% \usepackage{cite}
\usepackage{multicol}
\setlength{\columnsep}{1.5cm}
\setlength{\columnseprule}{.5pt}

\pagestyle{fancy}
\fancyhf{}
\rfoot{\textit{Página \thepage \hspace{1pt} de \pageref{LastPage}}}

%------ Paquetes matemáticos básicos --------%
\usepackage{amsmath}
\usepackage{amssymb}
\usepackage{amsthm}

%------ Paquetes para codigo --------%
\usepackage{tikz}

%------ Paquetes para citar --------%
\usepackage[spanish]{babel}
\usepackage{csquotes} % Recomendado para biblatex
% Carga biblatex con estilo APA; asegúrate de usar biber como backend.
\usepackage[backend=biber,style=apa]{biblatex}
\addbibresource{./src/referencias/referencias.bib} % Ruta a tu archivo .bib



\begin{document}

%------ Encabezado -------- %
\begin{center}
    \begin{minipage}{3cm}
    	\begin{center}
    		\includegraphics[height=3.2cm]{src/Img/Logo_UNAM.png}
    	\end{center}
    \end{minipage}\hfill
    \begin{minipage}{10cm}
    	\begin{center}
    	\textbf{\large Universidad Nacional Autónoma de México}\\[0.1cm]
        \textbf{Facultad de Ciencias}\\[0.1cm]
        \textbf{Inteligencia Artificial $|$ 7003}\\[0.1cm]
        Examen Parcial 1 $|$ Introducción y Agentes \\[0.1cm]
        Sosa Romo Juan Mario $|$ 320051926 \\[0.1cm]
        24/02/24
    	\end{center}
    \end{minipage}\hfill
    \begin{minipage}{3cm}
    	\begin{center}
    		\includegraphics[height=3.2cm]{src/Img/Logo_FC.png}
    	\end{center}
    \end{minipage}
\end{center}


\rule{16.9cm}{0.1mm}
\vspace{0.3cm}

\vspace{0.3cm}
%------ Ejercicios -------- %
\begin{enumerate}
    \item \textbf{Ejercicio 1.} (25 pts.) Evalúa las siguientes expresiones usando las téctnicas de paso de prámetros que se indican. En cada caso debes mostrar cómo queda el ambiente y memoria final, tal y como se vio en clase. \vspace{0.3cm}

\begin{itemize}
    \item Evalúa el siguiente código bajo paso por valor y por referencia.
\begin{lstlisting}
(let [(list1 (box '(1 2 3)))
    (list2 (box '(4 5 6)))
    (concat-lists (lambda (x y)
                    (begin
                        (set! x (append (unbox x) (unbox y)))
                        (set! y '(0)))))]
    (begin
        (concat-lists list1 list2)
        (list (unbox list1) (unbox list2))))
\end{lstlisting}

    \item Evalúa el siguiente código bajo paso por nombre y por necesidad.
\begin{lstlisting}
(let [(acc 0)
    (conditional (lambda (x)
                    (if (> x 0)
                        (begin (set! acc (+ acc 1)) acc)
                        (begin (set! acc (- acc 1)) acc))))
    (compute (lambda (y) (* y y)))]
(compute (conditional acc)))
\end{lstlisting}
\end{itemize}



    \item \textbf{Explica los tres paradigmas principales de la I.A., así como sus características principales (1 pt.).}

\begin{itemize}
    \item \textbf{Simbólico}
    
    En escencia este enfoque busca conseguir "inteligencia" mediante el uso de simbolos y reglas. Es útil si queremos hacer cosas mas deterministicas o mas estructurada. \vspace{.3cm}

    Es un enfoque mas limitado en terminos de escalabilidad, pues se tienen que de cierta forma saber de manera previa lo que se quiere que el sistema sepa. \vspace{.3cm}
    
    \cite{russell2020artificial}

    \item \textbf{Estadístico}
    
    Como su nombre lo dice, este paradigma se apoya de metodos probabilisticos y estadisticos como la regresión para intentar corregir o determinar la incertidumbre, incluyen metodos como los de aprendizaje supervisado, no supervisado y de refuerzo. \vspace{.3cm}

    En si son bastante utiles para ver patrones en grandes cantidades de datos y son lo que se implementa para modelos economicos o los conocidos algoritmos de redes sociales.\vspace{.3cm}

    En escencia este enfoque tiene 2 problemas principales, el primero y el mas grande (no solo aplica a este tipo de IA) es la cantidad y calidad de los datos, en si, al ser modelos hechos para procesar muchos datos, conseguirlos y ver que sean validos es complicado. El segundo problema es el de la caja negra, a diferencia del enfoque anterior en donde las cosas son relativamente mas simples de entender y simular, en este paradigma, al no necesitar mostrar o explicar los pasos si no mas bien el resultado y el modelo, se vuelve bastante mas complicado entender su funcionamiento. \vspace{.3cm}

    \cite{bishop2006pattern}

    \item \textbf{Neuronal}
    
    Finalmente, tenemos el paradigma neuronal, como su nombre lo indica, esta corriente se basa de como funcionan los cerebros organicos, especialmente el de los humanos; en su centro tiene la idea de computar a traves de una red de unidades independientes distribuídas que reciben una serie de señales y sacan otra serie de señales tras procesarlas. \vspace{.3cm}

    Ademas de las unidades que llaman neuronas, existen capas de entrada, ocultas y de salida, que son agrupaciónes de neuronas con comportamientos especificos. Una parte buena de este enfoque es que pierde gran parte de la estrucura que los otros 2 necesitan, volviendose mas flexible y permite enfocarse mas en las entradas y salidas, lo que a su vez significa que otra vez tenemos el problema de la caja negra y esta vez aun mas intenso. \vspace{.3cm}

    Este enfoque es el mas popular actualmente, a mi parecer por su gran escalabilidad, ademas, estos modelos son capaces de tratar con datos no estructurados, haciendolos mucho mejores en tareas de traduccion o de procesamiento de imágenes.\vspace{.3cm}

    \cite{goodfellow2016deep}
\end{itemize}

Finalmente, me gusta agregar que aunque son 3 enfoques diferentes, en la práctica es muy commun utilizar una combinación para aprovechar las ventajas de cada uno y cubrir lo que se puede de los otros.
    \item \textbf{Explica la diferencia entre I.A. débil e I.A. fuerte. Cita tus fuentes (0.5 pt.).}
    \item \textbf{Ejercicio 4.} (25 pts.) Realiza la inferencia de tipos vista en clase sobre la siguiente expresión, recuerda obtener las restricciones y usar el algoritmo de unificación para resolverlas. \vspace{0.3cm}

\begin{lstlisting}
((lambda (x) (* x 2)) (+ 2 3))
\end{lstlisting}

\textbf{Obtener las subexpresiones} \vspace{0.3cm}

\begin{itemize}
    \item 1. \hl{((lambda (x) (* x 2)) (+ 2 3))}
    \item 2. (\hl{(lambda (x) (* x 2))} (+ 2 3))
    \item 3. ((lambda \hl{(x)} (* x 2)) (+ 2 3))
    \item 4. ((lambda (x) \hl{(* x 2)}) (+ 2 3))
    \item 5. ((lambda (x) (* \hl{x} 2)) (+ 2 3))
    \item 6. ((lambda (x) (* x \hl{2})) (+ 2 3))
    \item 7. ((lambda (x) (* x 2)) \hl{(+ 2 3)})
    \item 8. ((lambda (x) (* x 2)) (+ \hl{2} 3))
    \item 9. ((lambda (x) (* x 2)) (+ 2 \hl{3}))
\end{itemize}

\textbf{Obtener las restricciones}
\begin{align*}
    [[2]] &= [[7]] \rightarrow [[1]] \\
    [[2]] &= [[3]] \rightarrow [[4]] \\
    [[3]] &= [[5]] \\ 
    [[4]] &= number \\
    [[5]] &= number \\
    [[6]] &= number \\
    [[7]] &= number \\
    [[8]] &= number \\
    [[9]] &= number
\end{align*}

\textbf{Sustituir en las restricciones (algoritmo de unificación)} \vspace{0.3cm}

\begin{center}
    \begin{tikzpicture}
        % Pila 1
        \foreach \y/\val in {
            0/$[[9]] = number$,
            1/$[[8]] = number$,
            2/$[[7]] = number$,
            3/$[[6]] = number$,
            4/$[[5]] = number$,
            5/$[[4]] = number$, 
            6/$[[3]] = [[5]]$, 
            7/$[[2]] = [[3]] \rightarrow [[4]]$, 
            8/$[[2]] = [[7]] \rightarrow [[1]]$
            } {
            \draw[thick] (-1.5, \y) rectangle (2.5, \y+1);
            \node at (0.5, \y+0.5) {\val};
        }
        \node[below] at (0.5, 0) {Pila de restricciones};
    
        % Pila 2
        \foreach \y/\val in {0/$\emptyset$} {
            \draw[thick] (4, \y) rectangle (7.5, \y+1);
            \node at (5.8, \y+0.5) {\val};
        }
        \node[below] at (5.8, 0) {Sustituciones};
    \end{tikzpicture}
\end{center}

\begin{center}
    \begin{tikzpicture}
        % Pila 1
        \foreach \y/\val in {
            0/$[[9]] = number$,
            1/$[[8]] = number$,
            2/$[[7]] = number$,
            3/$[[6]] = number$,
            4/$[[5]] = number$,
            5/$[[4]] = number$, 
            6/$[[3]] = [[5]]$, 
            7/$[[7]] \rightarrow [[1]] = [[3]] \rightarrow [[4]]$, 
            } {
            \draw[thick] (-1.8, \y) rectangle (2.8, \y+1);
            \node at (0.5, \y+0.5) {\val};
        }
        \node[below] at (0.5, 0) {Pila de restricciones};
    
        % Pila 2
        \foreach \y/\val in {0/$[[2]] := [[7]] \rightarrow [[1]]$} {
            \draw[thick] (4, \y) rectangle (7.5, \y+1);
            \node at (5.8, \y+0.5) {\val};
        }
        \node[below] at (5.8, 0) {Sustituciones};
    \end{tikzpicture}
\end{center}

\begin{center}
    \begin{tikzpicture}
        % Pila 1
        \foreach \y/\val in {
            0/$[[1]] = [[4]]$,
            1/$[[7]] = [[3]]$,
            2/$[[9]] = number$,
            3/$[[8]] = number$,
            4/$[[7]] = number$,
            5/$[[6]] = number$,
            6/$[[5]] = number$,
            7/$[[4]] = number$, 
            8/$[[3]] = [[5]]$,  
            } {
            \draw[thick] (-1.8, \y) rectangle (2.8, \y+1);
            \node at (0.5, \y+0.5) {\val};
        }
        \node[below] at (0.5, 0) {Pila de restricciones};
    
        % Pila 2
        \foreach \y/\val in {
            0/$[[2]] := [[7]] \rightarrow [[1]]$
        } {
            \draw[thick] (4, \y) rectangle (7.5, \y+1);
            \node at (5.8, \y+0.5) {\val};
        }
        \node[below] at (5.8, 0) {Sustituciones};
    \end{tikzpicture}
\end{center}

\begin{center}
    \begin{tikzpicture}
        % Pila 1
        \foreach \y/\val in {
            0/$[[1]] = [[4]]$,
            1/$[[7]] = [[5]]$,
            2/$[[9]] = number$,
            3/$[[8]] = number$,
            4/$[[7]] = number$,
            5/$[[6]] = number$,
            6/$[[5]] = number$,
            7/$[[4]] = number$, 
            } {
            \draw[thick] (-1.8, \y) rectangle (2.8, \y+1);
            \node at (0.5, \y+0.5) {\val};
        }
        \node[below] at (0.5, 0) {Pila de restricciones};
    
        % Pila 2
        \foreach \y/\val in {
            0/$[[2]] := [[7]] \rightarrow [[1]]$,
            1/$[[3]] := [[5]]$
        } {
            \draw[thick] (4, \y) rectangle (7.5, \y+1);
            \node at (5.8, \y+0.5) {\val};
        }
        \node[below] at (5.8, 0) {Sustituciones};
    \end{tikzpicture}
\end{center}

\begin{center}
    \begin{tikzpicture}
        % Pila 1
        \foreach \y/\val in {
            0/$[[1]] = number$,
            1/$[[7]] = [[5]]$,
            2/$[[9]] = number$,
            3/$[[8]] = number$,
            4/$[[7]] = number$,
            5/$[[6]] = number$,
            6/$[[5]] = number$,
            } {
            \draw[thick] (-1.8, \y) rectangle (2.8, \y+1);
            \node at (0.5, \y+0.5) {\val};
        }
        \node[below] at (0.5, 0) {Pila de restricciones};
    
        % Pila 2
        \foreach \y/\val in {
            0/$[[2]] := [[7]] \rightarrow [[1]]$,
            1/$[[3]] := [[5]]$,
            2/$[[4]] := number$
        } {
            \draw[thick] (4, \y) rectangle (7.5, \y+1);
            \node at (5.8, \y+0.5) {\val};
        }
        \node[below] at (5.8, 0) {Sustituciones};
    \end{tikzpicture}
\end{center}

\begin{center}
    \begin{tikzpicture}
        % Pila 1
        \foreach \y/\val in {
            0/$[[1]] = number$,
            1/$[[7]] = number$,
            2/$[[9]] = number$,
            3/$[[8]] = number$,
            4/$[[7]] = number$,
            5/$[[6]] = number$,
            } {
            \draw[thick] (-1.8, \y) rectangle (2.8, \y+1);
            \node at (0.5, \y+0.5) {\val};
        }
        \node[below] at (0.5, 0) {Pila de restricciones};
    
        % Pila 2
        \foreach \y/\val in {
            0/$[[2]] := [[7]] \rightarrow [[1]]$,
            1/$[[3]] := number$,
            2/$[[4]] := number$,
            3/$[[5]] := number$
        } {
            \draw[thick] (4, \y) rectangle (7.5, \y+1);
            \node at (5.8, \y+0.5) {\val};
        }
        \node[below] at (5.8, 0) {Sustituciones};
    \end{tikzpicture}
\end{center}

\begin{center}
    \begin{tikzpicture}
        % Pila 1
        \foreach \y/\val in {
            0/$[[1]] = number$,
            1/$[[7]] = number$,
            2/$[[9]] = number$,
            3/$[[8]] = number$,
            4/$[[7]] = number$,
            } {
            \draw[thick] (-1.8, \y) rectangle (2.8, \y+1);
            \node at (0.5, \y+0.5) {\val};
        }
        \node[below] at (0.5, 0) {Pila de restricciones};
    
        % Pila 2
        \foreach \y/\val in {
            0/$[[2]] := [[7]] \rightarrow [[1]]$,
            1/$[[3]] := number$,
            2/$[[4]] := number$,
            3/$[[5]] := number$,
            4/$[[6]] := number$
        } {
            \draw[thick] (4, \y) rectangle (7.5, \y+1);
            \node at (5.8, \y+0.5) {\val};
        }
        \node[below] at (5.8, 0) {Sustituciones};
    \end{tikzpicture}
\end{center}

\begin{center}
    \begin{tikzpicture}
        % Pila 1
        \foreach \y/\val in {
            0/$[[1]] = number$,
            1/$[[7]] = number$,
            2/$[[9]] = number$,
            3/$[[8]] = number$,
            } {
            \draw[thick] (-1.8, \y) rectangle (2.8, \y+1);
            \node at (0.5, \y+0.5) {\val};
        }
        \node[below] at (0.5, 0) {Pila de restricciones};
    
        % Pila 2
        \foreach \y/\val in {
            0/$[[2]] := number \rightarrow [[1]]$,
            1/$[[3]] := number$,
            2/$[[4]] := number$,
            3/$[[5]] := number$,
            4/$[[6]] := number$,
            5/$[[7]] := number$
        } {
            \draw[thick] (3.8, \y) rectangle (7.8, \y+1);
            \node at (5.8, \y+0.5) {\val};
        }
        \node[below] at (5.8, 0) {Sustituciones};
    \end{tikzpicture}
\end{center}

\begin{center}
    \begin{tikzpicture}
        % Pila 1
        \foreach \y/\val in {
            0/$[[1]] = number$,
            1/$[[7]] = number$,
            2/$[[9]] = number$,
            } {
            \draw[thick] (-1.8, \y) rectangle (2.8, \y+1);
            \node at (0.5, \y+0.5) {\val};
        }
        \node[below] at (0.5, 0) {Pila de restricciones};
    
        % Pila 2
        \foreach \y/\val in {
            0/$[[2]] := number \rightarrow [[1]]$,
            1/$[[3]] := number$,
            2/$[[4]] := number$,
            3/$[[5]] := number$,
            4/$[[6]] := number$,
            5/$[[7]] := number$,
            6/$[[8]] := number$
        } {
            \draw[thick] (3.8, \y) rectangle (7.8, \y+1);
            \node at (5.8, \y+0.5) {\val};
        }
        \node[below] at (5.8, 0) {Sustituciones};
    \end{tikzpicture}
\end{center}

\begin{center}
    \begin{tikzpicture}
        % Pila 1
        \foreach \y/\val in {
            0/$[[1]] = number$,
            1/$[[7]] = number$,
            } {
            \draw[thick] (-1.8, \y) rectangle (2.8, \y+1);
            \node at (0.5, \y+0.5) {\val};
        }
        \node[below] at (0.5, 0) {Pila de restricciones};
    
        % Pila 2
        \foreach \y/\val in {
            0/$[[2]] := number \rightarrow [[1]]$,
            1/$[[3]] := number$,
            2/$[[4]] := number$,
            3/$[[5]] := number$,
            4/$[[6]] := number$,
            5/$[[7]] := number$,
            6/$[[8]] := number$,
            7/$[[9]] := number$
        } {
            \draw[thick] (3.8, \y) rectangle (7.8, \y+1);
            \node at (5.8, \y+0.5) {\val};
        }
        \node[below] at (5.8, 0) {Sustituciones};
    \end{tikzpicture}
\end{center}

\begin{center}
    \begin{tikzpicture}
        % Pila 1
        \foreach \y/\val in {
            0/$[[1]] = number$,
            } {
            \draw[thick] (-1.8, \y) rectangle (2.8, \y+1);
            \node at (0.5, \y+0.5) {\val};
        }
        \node[below] at (0.5, 0) {Pila de restricciones};
    
        % Pila 2
        \foreach \y/\val in {
            0/$[[2]] := number \rightarrow [[1]]$,
            1/$[[3]] := number$,
            2/$[[4]] := number$,
            3/$[[5]] := number$,
            4/$[[6]] := number$,
            5/$[[7]] := number$,
            6/$[[8]] := number$,
            7/$[[9]] := number$,
            8/$[[7]] := number$,
        } {
            \draw[thick] (3.8, \y) rectangle (7.8, \y+1);
            \node at (5.8, \y+0.5) {\val};
        }
        \node[below] at (5.8, 0) {Sustituciones};
    \end{tikzpicture}
\end{center}

\begin{center}
    \begin{tikzpicture}
        % Pila 1
        \foreach \y/\val in {
            0/$\emptyset$,
            } {
            \draw[thick] (-1.8, \y) rectangle (2.8, \y+1);
            \node at (0.5, \y+0.5) {\val};
        }
        \node[below] at (0.5, 0) {Pila de restricciones};
    
        % Pila 2
        \foreach \y/\val in {
            0/$[[2]] := number \rightarrow number$,
            1/$[[3]] := number$,
            2/$[[4]] := number$,
            3/$[[5]] := number$,
            4/$[[6]] := number$,
            5/$[[7]] := number$,
            6/$[[8]] := number$,
            7/$[[9]] := number$,
            8/$[[7]] := number$,
            9/$[[1]] := number$
        } {
            \draw[thick] (3.4, \y) rectangle (8.1, \y+1);
            \node at (5.8, \y+0.5) {\val};
        }
        \node[below] at (5.8, 0) {Sustituciones};
    \end{tikzpicture}
\end{center}

Que tiene sentido pues la expresión original era una aplicación que usaba la función que va de $number \rightarrow number$ y el argumento es una suma de tipo $number$ el cuerpo de la función nos regresa efectivamente un number siempre y cuando x sea un número. \vspace{0.3cm}
    \item \textbf{Investiga en qué informe se declara el fracaso del programa de traducción de máquina, y explica brevemente las razones que en él se esbozan. Cita tus fuentes (1 pt.).}
    \item \textbf{Explica brevemente qué es un agente racional en el contexto de inteligencia artificial. Preferiblementeincluye un diagrama en tu descripción (1 pt.).}
    \item \textbf{Menciona dos diferencias entre la función de rendimiento y el programa del agente (1 pt.).} \vspace{.3cm}

Como ya sabemos, la función de rendimiento busca evaluar qué tan bien un agente cumple con sus objetivos en su entorno, mientras que el programa del agente se encarga de implementar este comportamiento. Por lo tanto, esta primera diferencia puede compararse con la relación entre teoría y práctica.  

De la misma manera, es evidente que la implementación de la función de rendimiento es independiente del programa del agente; en este sentido, la función representa el \textit{qué} y el programa el \textit{cómo}.  

Finalmente, cabe mencionar que es relativamente más fácil cambiar el programa del agente para mejorar su desempeño, mientras que modificar la función de rendimiento suele ser mucho más complicado. \vspace{.3cm}

    \item \textbf{Considera el mundo de la aspiradora constituido por dos celdas. Si el agente utiliza el siguiente algoritmo para desempeñar sus funciones:}

\includegraphics[width=16cm]{src/Img/Screenshot_20250224_024900.png}

indica el tipo de agente en el que podría clasificarse (1 pt.). \vspace{.3cm}

Me parece que la aspiradora es un \textbf{agente reactivo simple}, ya que se basa únicamente en su percepción actual, ignorando lo que haya ocurrido antes o lo que pueda suceder después. Se observa que utiliza condiciones muy simples diseñadas únicamente para evaluar la celda en la que se encuentra. \vspace{.2cm}

Además, considerando que este ambiente solo tiene dos posibles estados de suciedad y que las únicas acciones disponibles son moverse de casilla o limpiar, este modelo es bastante adecuado y debería cumplir su función de manera más o menos efectiva (nótese que si no hay suciedad, la aspiradora estará en un ciclo constante de movimiento de un lado a otro).  

    \item \textbf{Menciona tres categorías en las que podemos clasificar los entornos de trabajo (0.5 pt.).} \vspace{.3cm}

Las dimensiones que~\cite{SotoAstorga2025} menciona para categorizar los entornos según sus propiedades son:

\begin{itemize}
    \item \textbf{Observabilidad:} Se refiere a la cantidad de información que el agente puede obtener de su entorno mediante sus sensores. 
    \item \textbf{Cantidad de agentes:} Considera si existen otros agentes en el ambiente y cómo interactúan con nuestro agente.
    \item \textbf{Determinismo:} Un entorno es determinista si las acciones del agente, junto con el estado previo, determinan completamente la evolución del ambiente. De lo contrario, el entorno es estocástico o estratégico si los cambios dependen de otros agentes.
    \item \textbf{Episodicidad:} Un entorno es episódico si cada decisión puede considerarse de manera independiente, mientras que es secuencial cuando las acciones presentes afectan las decisiones futuras.
    \item \textbf{Estaticidad:} Evalúa qué tanto cambia el entorno mientras el agente ejecuta sus acciones. 
    \item \textbf{Continuidad:} Distingue si las acciones y el entorno evolucionan en intervalos de tiempo preestablecidos (discreto) o si se consideran continuos, como en el mundo real.
\end{itemize}

    \item \textbf{Describe dos situaciones en las que el agente basado en utilidad tiene ventaja sobre el agente basado en objetivo (0.5 pt.).} \vspace{.3cm}

Como menciona~\cite{SotoAstorga2025}, aunque los agentes basados en objetivos son más flexibles que los reactivos simples o con modelo, cuando los objetivos no están bien definidos o no son completamente alcanzables, pueden producir resultados subóptimos. \vspace{.2cm}

Por ejemplo, si quisiéramos diseñar un taxi con IA, este tendría al menos dos objetivos que podrían ser contradictorios entre sí: llegar rápido y llegar seguro. Para un agente basado en objetivos, la contradicción entre estos podría dificultar la toma de decisiones. En cambio, un agente basado en utilidad podría evaluar múltiples rutas y seleccionar aquella que mejor equilibre ambos factores. \vspace{.2cm}

Por lo tanto, en entornos dinámicos o cuando se manejan múltiples objetivos complejos, un agente basado en utilidad ofrece mayor granularidad y precisión en la toma de decisiones en comparación con un agente basado en objetivos.
    \item \textbf{Explica la noción de aprendizaje para agentes (1 pt.).}
    \item \textbf{Para el mundo de un robot de servicio de cafetería, enuncia: sus sensores, sus efectores, su ambiente, su entorno de trabajo. Además, propón una medida de rendimiento para que su agencia sea racional. Además, indica qué tipo de agente sería mejor implementar para su servicio: basado en modelo, que aprende, dirigido por tabla, o reactivo simple (1 pt.).}
\end{enumerate}

\printbibliography


\end{document}